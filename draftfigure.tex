% \draftfigure --%
%            modifying drafted figures
% Copyright (c) 2017 Lukas C. Bossert
%  
% This work may be distributed and/or modified under the
% conditions of the LaTeX Project Public License, either version 1.3
% of this license or (at your option) any later version.
% The latest version of this license is in
%   http://www.latex-project.org/lppl.txt
% and version 1.3 or later is part of all distributions of LaTeX
% version 2005/12/01 or later.
% !TEX program = lualatex
\documentclass[a4paper,
12pt,
english
]{ltxdoc}


\usepackage{iftex}
\ifPDFTeX
  \usepackage[utf8]{inputenc}
  \usepackage[T1]{fontenc}
  \usepackage{lmodern}
\else
    \ifXeTeX
    \usepackage{fontspec}
		\def\compiler{\hologo{XeLaTeX}}
  \else 
	  \usepackage{fontspec}
    \usepackage{luatextra}
    \usepackage{luaotfload}
	\def\compiler{\hologo{LuaLaTeX}}
\fi
  \defaultfontfeatures{Ligatures=TeX}
\fi
\listfiles
 \usepackage[oldstyle]{libertine}
\defaultfontfeatures[AnonymousPro]
  {
    Extension      = .ttf                       ,
    BoldFont       = AnonymousPro-Bold          ,
    ItalicFont     = AnonymousPro-BoldItalic    ,
    BoldItalicFont = AnonymousPro-Italic        ,
    UprightFont    = AnonymousPro-Regular       ,
  }
\setmonofont[Scale= MatchUppercase]{AnonymousPro}



\usepackage{xr-hyper}
\usepackage{url}
\usepackage{xspace}


\usepackage[
	backend=biber,
]{biblatex}
\usepackage{metalogo}
\usepackage{hologo}
\usepackage{babel}
\usepackage{coolthms}


\usepackage{chngcntr}

\usepackage[
  autostyle=true,%
]{csquotes}
\usepackage{multicol}
  \setlength{\columnsep}{1.5cm}
  \setlength{\columnseprule}{0.2pt}

\usepackage{xcolor}
\definecolor{codeblue}{RGB}{0,65,137}
\definecolor{codegreen}{RGB}{147,193,26}
\definecolor{codegray}{rgb}{0.5,0.5,0.5}
\definecolor{codepurple}{rgb}{0.58,0,0.82}
\definecolor{backcolour}{rgb}{0.95,0.95,0.92}


\usepackage[ 
	headsepline, 
	footsepline,
%	plainfootsepline, 
%markcase=upper, 
automark, 
draft=false,
]{scrlayer-scrpage} 
\pagestyle{scrheadings}
\clearscrheadfoot
\ihead{\normalfont\footnotesize \texttt{bib}\LaTeX-style \texttt{draftfigure \draftfigureversion} \copyright\ by Lukas C. Bossert}%
\rofoot{\normalfont\footnotesize  \textbf{\sffamily \thepage}}
\lofoot{\normalfont\footnotesize  \href{http://www.digitales-altertum.de}{digitales-altertum.de}}

\usepackage[%
%  flushmargin, %
%  marginal,
  ragged,%
  hang, %
  bottom%
]{footmisc}



\usepackage{enumitem}
\setlength{\parindent}{0pt}
\setlength{\parskip}{6pt plus 2pt minus 2pt}
\setenumerate[1]{label=(\alph*),leftmargin=*,nolistsep,parsep=\parskip}
\usepackage{changepage}
\makeindex

\usepackage{caption}

\usepackage[%
  skins,%
  listings,%
  breakable,%
]{tcolorbox}
\lstMakeShortInline{|}

\newtcblisting[
  auto counter,
  list inside=bibexample,
%  number within=subsection,
  crefname={Example}{Examples}
]{bibexample}[2][]{%
  listing only,
  breakable,
  top=0.5pt,
  bottom=0.5pt,
  colback=codegreen!10,
  colframe=codegreen,
    left=5pt,
    right=5pt,
    sharp corners,
  boxrule=0pt,
  bottomrule=2pt,
  toprule=2pt,
  enhanced jigsaw,
  listing options={%style=tcblatex,
    numbers=left,
    numberstyle=\small\color{codeblue},
    moredelim={[is][keywordstyle]{@@}{@@}},
        basicstyle=\small\ttfamily,
    breaklines=true,
    breakautoindent=false,
    breakindent=0pt,
    escapeinside={{*@}{@*}},
  },%
  lefttitle=0pt,
  coltitle=black,
  colbacktitle=codegreen!20,
  fonttitle=\bfseries\sffamily\footnotesize,
  title={Example \thetcbcounter:  #2}, 
  #1,%  
  borderline north={1pt}{14.4pt}{codegreen,dashed},
}

\newtcblisting[
auto counter,
]{code}{%
    listing only,
    breakable,
    top=0.2pt,
    bottom=0.2pt,
    colback=codegreen!10,
    colframe=codegreen,
    left=5pt,
    right=5pt,
      sharp corners,
    boxrule=0pt,
    bottomrule=0pt,
    toprule=0pt,
    enhanced jigsaw,
    listing options={%style=tcblatex,
%        numbers=left,
        numberstyle=\tiny\color{codeblue},
        moredelim={[is][keywordstyle]{@@}{@@}},
        basicstyle=\small\ttfamily,
        breaklines=true,
        breakautoindent=false,
        breakindent=0pt,
        escapeinside={{*@}{@*}},
    },%
    lefttitle=0pt,
    coltitle=codeblue,
    colbacktitle=codegreen!10,
%    fonttitle=\bfseries\footnotesize,
%    title={Example \thetcbcounter:  #2}, 
%   #1,%  
%    borderline north={1pt}{14.4pt}{codegreen,dashed},
}

\newtcolorbox{bibbox}[1]{
      breakable,
      top=5pt,
      bottom=5pt,
      colback=codeblue!10,
      colframe=codeblue,
      left=5pt,
      right=5pt,
        sharp corners,
      boxrule=0pt,
      bottomrule=2pt,
      toprule=2pt,
      enhanced jigsaw,
        lefttitle=0pt,
        coltitle=black,
          fontupper=\small,%\ttfamily,
        colbacktitle=codeblue!20,
        fonttitle=\bfseries\footnotesize,
  title={\Cref{#1}},
        borderline north={1pt}{14.4pt}{codeblue,dashed},
}


\newtcolorbox{marker}[1][]{
enhanced,
  before skip=2mm,after skip=3mm,
  boxrule=0.4pt,left=5mm,right=2mm,top=1mm,bottom=1mm,
  colback=backcolour,
  colframe=yellow!20!black,
  sharp corners,rounded corners=southeast,arc is angular,arc=3mm,
  underlay={%
    \path[fill=tcbcol@back!80!black] ([yshift=3mm]interior.south east)--++(-0.4,-0.1)--++(0.1,-0.2);
    \path[draw=tcbcol@frame,shorten <=-0.05mm,shorten >=-0.05mm] ([yshift=3mm]interior.south east)--++(-0.4,-0.1)--++(0.1,-0.2);
    \path[fill=red!50!black,draw=none] (interior.south west) rectangle node[white]{\Huge\bfseries !} ([xshift=4mm]interior.north west);
    },
  drop fuzzy shadow,#1
  }
  
\newtcolorbox{examplebox}[1][]{
examplebox,
}

\tcbset{examplebox/.style={%
              boxrule=0pt,
              bottomrule=2pt,
              toprule=2pt,
 colframe=codeblue,
  colback=codegreen!10,
   coltitle=codegreen!10,%  coltitle=codeblue,
  bicolor,
      sharp corners,
 fontupper=\small\ttfamily,
  colbacklower=codeblue!10,
  fonttitle=\sffamily\bfseries,
  }}



\newtcblisting{example}{%
    before skip=\baselineskip,
examplebox,
breakable,
%  sidebyside,
listing and text,
}




\usepackage{hyperxmp}
\usepackage{hyperref}
\hypersetup{					% setup the hyperref-package options
  unicode       = true,
	pdftitle      = {draftfigure},	% 	- title (PDF meta)
	pdfsubject    = {},% 	- subject (PDF meta)
	pdfauthor      = {Lukas C. Bossert},	% 	- author (PDF meta)
	pdfauthortitle = {},
	pdfcopyright   = {This work may be distributed and/or modified under the
                    conditions of the LaTeX Project Public License, either version 1.3
                    of this license or (at your option) any later version.},
	pdfhighlight   = /N,
	pdfdisplaydoctitle = true,
	pdfdate        = {\the\year-\the\month-\the\day}
	pdflang        = {en},
	pdfcaptionwriter = {Lukas C. Bossert},
	pdfkeywords    = {},
	pdfproducer={LuaLaTeX},
	pdflicenseurl  = {http://www.latex-project.org/lppl.txt},
	plainpages     = false,			% 	- 
  colorlinks     = true, %Colours links instead of ugly boxes
  urlcolor       = codeblue, %Colour for external hyperlinks
  linkcolor      = codeblue, %Colour of internal links
  citecolor      = black, %Colour of citations
  linktoc        = page,
  pdfborder      = {0 0 0},			% 	-
	breaklinks     = true,			% 	- allow line break inside links
	bookmarksnumbered  = true,		%
	bookmarksopenlevel = 4,
	bookmarksopen  = true,		%
	final          = true
}
\usepackage{bookmark}

\crefformat{lstlisting}{#2example\ #1#3}
\Crefformat{lstlisting}{#2Example #1#3}
\crefmultiformat{lstlisting}{#2examples #1#3}{; #2#1#3}{; #2#1#3}{; #2#1#3}
\Crefmultiformat{lstlisting}{#2Examples #1#3}{; #2#1#3}{; #2#1#3}{; #2#1#3}
\Crefrangeformat{lstlisting}{#3Examples #1#4--#5#2#6}
\crefrangeformat{lstlisting}{#3examples #1#4--#5#2#6}

\newcommand\df{draftfigure\xspace}
\newcommand\dfstring{|draftfigure|\xspace}
\setkeys{Gin}{width=.3\linewidth}

\begin{document}
\title{\texttt{\df} -- \\ modifying drafted figures\footnote{%
For further information about the code visit \href{https://github.com/LukasCBossert/draftfigure}{https://github.com/LukasCBossert/draftfigure}: 
Comments and criticisms are welcome.}}
\author{Lukas C. Bossert\\{\small \href{mailto:lukas@digitales-altertum.de}{lukas@digitales-altertum.de}}}
\date{Version: \dfdate{} (\dfversion)} 
\maketitle

\begin{abstract}
\noindent This package allows you to customize the outcome of drafted figures.
\end{abstract}


%\begin{multicols}{1}
\footnotesize\parskip=0mm \tableofcontents
%\end{multicols}

\section{Installation}
\dfstring is part of the distributions MiK\TeX \footnote{Website: \url{http://www.miktex.org}.} 
and \TeX Live\footnote{Website: \url{http://www.tug.org/texlive}.}~-- thus, you
can easily install it using the respective package manager. 
If you would like to
install \dfstring manually, do the following:
Download the folder \dfstring with all relevant files from the CTAN-server\footnote{\url{https://www.ctan.org/pkg/\df}} and copy the content of the |zip|-file to the \texttt{\$LOCALTEXMF} directory of
 your system.\footnote{If you don't know what that is, have a look at
\url{http://www.tex.ac.uk/cgi-bin/texfaq2html?label=tds} or 
\url{http://mirror.ctan.org/tds/tds.html}.} 
Refresh your filename database.\footnote{ 
Here is some additional information from the UK \TeX\ FAQ:
\begin{itemize}[nosep,after=\vspace{-\baselineskip} ]
  \item \href{%
    http://www.tex.ac.uk/cgi-bin/texfaq2html?label=install-where}{%
    Where to install packages}
  \item \href{%
    http://www.tex.ac.uk/cgi-bin/texfaq2html?label=inst-wlcf}{%
    Installing files \enquote{where \LaTeX /TeX\ can find them}}
  \item \href{%
    http://www.tex.ac.uk/cgi-bin/texfaq2html?label=privinst}{%
    \enquote{Private} installations of files}
\end{itemize}
}
%%introduction from biblatex-dw copied and applied. might to be rewritten.

\clearpage
\section{Status Quo}

Usually you see all of your images in your PDF:
\begin{examplebox}
\begin{figure}[H]
  \centering
  \includegraphics{example-image-a}
  \caption{A}
  \includegraphics{example-image-b}
  \caption{B}
  \includegraphics{example-image-c}
  \caption{C}
\end{figure}
\end{examplebox}
Sometimes you do not need to see the images or want to omit them due to faster compiling or you do not have the right to publish them.


You can \enquote{switch off} the images by using the option |draft| either with 
|\documentclass[draft]|\marg{class} 
%\cs{documentclass}\oarg{draft}\marg{clss} 
or with 
|\usepackage[draft]{graphicx}|.
%\cs{usepackage}\oarg{draft}\marg{graphicx}


If you do not like the outcome with the filename and its path written with |\ttfamily| then use \dfstring.
\clearpage

\section{Usage}
 \DescribeMacro{draftfigure}  The name of the package is  \dfstring which has to be activated in the preamble. 

\begin{code}
\usepackage[%
  *@\meta{further options}@*
  ]{draftfigure}
\end{code}
Without any further  Now you will now get white boxes instead of the images.
\begin{examplebox}
\begin{figure}[H]\setkeys{Gin}{draft=true}
  \centering
  \includegraphics{example-image-a}
  \caption{A}
  \includegraphics{example-image-b}
  \caption{B}
  \includegraphics{example-image-c}
  \caption{C}
\end{figure}
\end{examplebox}
\clearpage

\section{Options}

\subsection{Preamble}
\subsubsection{style}
\subsubsection{size}
\subsubsection{position}
\subsubsection{allfiguresdraft}
\subsubsection{content}
\subsubsection{filename}
\subsubsection{noframe}

\subsection{Individually possible}

You can also set |draft| to individual images.
\begin{codeexample}
\begin{figure}[H]
  \centering
  \includegraphics[draft]{example-image-a}
  \caption{A}
  \includegraphics{example-image-b}
  \caption{B}
  \includegraphics[draft]{example-image-c}
  \caption{C}
\end{figure}
\end{codeexample}

\subsubsection{style}
\subsubsection{size}
\subsubsection{position}
\subsubsection{content}
\subsubsection{filename}
\clearpage

\section{Things to work on}

- noframe for individual image\\
- combine different styles: sc and sf etc.\\
- color frame\\
- image instead of text in background\\

\end{document}